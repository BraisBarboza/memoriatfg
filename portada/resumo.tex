%%%%%%%%%%%%%%%%%%%%%%%%%%%%%%%%%%%%%%%%%%%%%%%%%%%%%%%%%%%%%%%%%%%%%%%%%%%%%%%%

\pagestyle{empty}
\begin{abstract}
  La imagen médica se ha beneficiado a lo largo del tiempo de los avances en las técnicas de visualización de contenido con el fin de poder brindar una mejor comprensión de los datos mostrados que se traduzca en una mejor atención médica. Es por ello que en este trabajo se analizan métodos para combinar la imagen médica con la realidad aumentada, otro campo cuyo auge en los últimos años para aplicaciones tanto industriales como de ocio no han pasado desapercibidas. Se lleva a cabo un análisis completo para proporcionar una solución funcional, desde la segmentación de los modelos desde una Tomografía Computerizada (TC), el desarrollo de un marcador para el seguimiento y la alineación en 3D hasta la implementación de una solución para llevarlo a cabo. Con el objetivo de servir como discusión de futuros desarrollos, se concibe teniendo como piedra angular su libre disposición y el software libre como alternativa a métodos existentes tras licencias restrictivas.

  \vspace*{25pt}
  \begin{segundoresumo}
    Medical imaging has benefited over time from advances in content visualization techniques in order to provide a better understanding of the displayed data, leading to improved healthcare. Therefore, this project examines methods for combining medical imaging with augmented reality, another field whose recent rise for both industrial and recreational applications has not gone unnoticed. A comprehensive analysis is carried out to provide a functional solution, including model segmentation from a computerized tomography (CT) scan, marker development for tracking and 3D alignment, and the implementation of a solution to execute the process. In order to serve as a discussion of future developments, this work is conceived with the cornerstone of open availability and open-source software as an alternative to existing methods with restrictive licenses.
  \end{segundoresumo}
  \vspace*{25pt}
  \begin{multicols}{2}
\begin{description}
\item [\palabraschaveprincipal:] \mbox{} \\[-20pt]
  \begin{itemize}
    \item Realidad aumentada
    \item Impresion 3D
    \item Imagen médica
  \end{itemize}
\end{description}
\begin{description}
\item [\palabraschavesecundaria:] \mbox{} \\[-20pt]
  \begin{itemize}
    \item Augmented reality
    \item 3D Printing
    \item Medical imaging
  \end{itemize}
\end{description}
\end{multicols}

\end{abstract}
% \textbf{Herramientas y tecnologías empleadas:}
% \begin{itemize}
%     \item Hardware:
%     \begin{itemize}
%         \item Desarrollo: Portátil Intel core i7 @2.5GHz, 16 GB RAM, Nvidia Geforce 1050.
%         \item Pruebas: PC 12th Gen Intel(R) Core(TM) i9-12900KF   3.20 GHz, 64,0 GB RAM, Nvidia RTX 3090.
%     \end{itemize}
%     \item Software:
%         \begin{itemize}
%             \item S.O: Windows 11.
%             \item Lenguajes: C++.
%             \item Librerías: OpenCV, OpenXR.
%             \item Compilación: Cmake
%             \item Otras Herramientas: ViveSR Works
%         \end{itemize}
% \end{itemize}
\pagestyle{fancy}

%%%%%%%%%%%%%%%%%%%%%%%%%%%%%%%%%%%%%%%%%%%%%%%%%%%%%%%%%%%%%%%%%%%%%%%%%%%%%%%%
