\chapter{Visualización}
\label{chap:visualización}
\lettrine{U}{no} de los principales desafíos del proyecto fue la implementación de un sistema que permitiera la obtención del vídeo procedente de las cámaras a partir del cual poder extraer los datos necesarios para el seguimiento del marcador fiduciario, y posteriormente la síntesis de las imágenes que serán mostradas por el dispositivo de realidad aumentada.
El mundo de la realidad virtual se encuentra en continuo desarrollo. Dado que es un sector en el que en la actualidad el factor limitante es el propio hardware, una gran cantidad de fabricantes han implementado sus propios sistemas de \acrfull{xr}. Esto se traduce en la necesidad de implementar un estándard para el desarrollo multiplataforma, que debe ser altamente versátil para soportar todos los formatos de dispositivo (\acrshort{hmd}, smartphones, sistemas compuestos por componentes personalizados, etc \dots). En este caso la piedra angular que se usó para solventar este problema fue \href{https://www.khronos.org/api/index_2017/openxr}{OpenXR}.

\subsubsection{ViveSRWorks}

